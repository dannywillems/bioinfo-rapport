%!TEX root=rapport.tex
\section{Organisation du travail}

Lors de la réalisation de ce travail, les tâches réalisées par chacun sont les suivantes:\\
Aline a :
\begin{itemize}
	\item[$\bullet$]Implémenté la lecture et l'écriture de fichiers fasta;
	\item[$\bullet$]Implémenté l'algorithme d'alignement semi-global;
	\item[$\bullet$]Réfléchi quant à la meilleure manière de représenter les ensembles liés à chaque fragment pour l'algorithme Greedy.
	 Cette réflexion a mené à l'implémentation de la structure de données \verb|Union-Find|;
	\item[$\bullet$]Implémenté l'algorithme Greedy et multi-threadé le calcul d'arcs nécéssaires pour la réalisation de l'algorithme Greedy, une fois que celui-ci semblait suffisamment stable.
\end{itemize}
$ $\\
Danny a:
\begin{itemize}
	\item[$\bullet$] Implémenté l'alignement des séquences à partir du chemin hamiltonien;
	\item[$\bullet$] Implémenté le consensus final;
	\item[$\bullet$] Parsé la ligne de commande et écrit le build.xml pour
		simplifier la création des archives zip et jar.
	\item[$\bullet$] Implémenté la classe \verb|SequenceAbstract| pour
		représenter de manière différente les séquences pour l'alignement
		(voir \ref{subsubsection:repr_sequence_alignment}).
\end{itemize}

Nous avons également eu une discussion préalable afin de réfléchir quant au meilleur moyen de stocker une séquence d'ADN, plus tard Aline a spécifié par quel type d'objet serait représenté le chemin hamiltonien permettant de retrouver la séquence finale et enfin nous nous sommes interrogés ensemble quant à notre compréhension commune de la propagation des gaps pour l'alignement des séquences.
\begin{comment}
De plus, Aline ayant également réfléchi à la manière d'obtenir la séquence finale à partir du chemin hamiltonien donné par l'algorithme greedy avant que Danny ne propose sa propre solution, elle a également exposé son point de vue quant à la réalisation de cette dernière étape. Le temps lui ayant manqué, son algorithme n'est pas totalement abouti et nous avons donc continué de travailler à partir de celui de Danny.
\end{comment}

Enfin, quant à la réalisation de ce rapport, nous l'avons réalisé à deux, chacun
se concentrant surtout sur l'explication des parties du travail qu'il a
réalisées.
