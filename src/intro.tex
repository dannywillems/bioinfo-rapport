%!TEX root=rapport.tex

\section{Introduction}

En biologie, il arrive qu'à partir d'un ensemble de fragments d'ADN on désire retrouver la séquence initiale correspondant à ces fragments.
C'est la tâche qui nous a été demandée. 

Le principe ne se résume pas simplement à prendre chaque fragment et à les mettre bout à bout afin de construire une super-chaîne. En effet,
comme les fragments ne sont pas coupés \og bout à bout\fg~ils se recouvrent partiellement. De plus, il se peut que lors de la fragmentation de l'ADN des nucléotides se soient rajoutés ou au contraire aient disparu (indel). De plus, si les deux brins d'ADN ont été fragmentés en même temps, il faut tenir compte du fait qu'un des brins est dans le sens \textbf{5'-3'} tandis que l'autre est dans le sens \textbf{3'-5'} et correspond au complémentaire du premier.

Notre rapport est rédigé de la manière suivante: dans la section suivante nous expliquons comment nous avons implémenté une solution à ce problème. Nous commençons par en expliquer brièvement chaque étape et ensuite nous expliquons plus précisément chacune d'elles. Ensuite, nous exposons et interprétons les résultats obtenus. Enfin, nous explicitons les forces et faiblesses que, d'après nous, notre implémentation comporte.
