%!TEX root=rapport.tex

\section{Introduction}

En biologie, il est très utile d'aligner des séquences d'ADN. En effet, lors du séquençage de l'ADN, un ensemble de fragments d'ADN  est récupéré et on désire retrouver la séquence initiale correspondant à ces fragments. C'est la tâche qui nous a été demandée. 

Ce processus est un un processus compliqué. Il ne se résume pas simplement à prendre chaque fragment et à les mettre bout à bout afin de construire une super-chaîne. En effet,
lors du séquençage, il est impossible d'extraire des fragments en des positions bien précises. De ce fait, ils se recouvrent partiellement. De plus, il se peut que lors de la fragmentation de l'ADN, des nucléotides se soient rajoutés ou au contraire aient disparu (indel) ou encore que des substitutions aient eu lieu. Enfin, une molécule d'ADN est composée de deux brins différents: un premier brin qui a une orientation \textbf{5'-3'} et un second - qui correspond à son complémentaire- qui a une orientation \textbf{3'-5'}. Le processus de séquençage ne permet pas non plus de connaître quelle est l'orientation du fragment.

Notre rapport est rédigé de la manière suivante: dans la section suivante nous expliquons comment nous avons implémenté une solution à ce problème. Nous commençons par en expliquer brièvement chaque étape et ensuite nous expliquons plus précisément chacune d'elle. Ensuite, nous exposons et interprétons les résultats obtenus. Enfin, nous expliquons la répartition des tâches et nous terminons en  explicitant les forces et faiblesses que, d'après nous, notre implémentation comporte.
